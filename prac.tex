\documentclass[12pt]{article}
\usepackage[margin=1in]{geometry}
\usepackage{amsmath,amssymb,graphicx}
\usepackage{listings}
\usepackage{xcolor}
\lstset{
  breaklines=true,
  basicstyle=\small\ttfamily,
  columns=fullflexible,
  frame=single,
  breakatwhitespace=true,
  keywordstyle=\color{blue}\bfseries,
  commentstyle=\color{gray},
  stringstyle=\color{orange}
}

\begin{document}
\title{CPSC 223 Spring 2025\\Practice Midterm \#1}
\author{}
\date{\today}
\maketitle

\noindent\textbf{Instructions:} Write your name and NetID on your answer sheet. You may use one 8½×11" crib sheet (both sides). No electronic devices. Show all work for partial credit. For multiple-choice questions, fill in the box.

\vspace{1em}

%% Problem 1
\section*{Problem 1: Basics (15 points)}
\begin{enumerate}
  \item[(a)] (3 pts) Given the files below, mark an X next to each file that contains the machine code for the function \texttt{process\_data} after running \texttt{make}.\\
  \texttt{dataio.h}\quad \texttt{dataio.c}\quad \texttt{dataio.o}\quad \texttt{main.c}\\
  \texttt{main.o}\quad \texttt{process\_data.c}\quad \texttt{utils.c}\quad \texttt{utils.h}
  \\

  \item[(b)] (5 pts) Complete the code below: fill blanks (1)--(4) with letters (A)--(F).\\
  \begin{lstlisting}[language=C]
typedef struct {
  char *name;
  int  score;
} Player;

Player *p = NULL;
/* 1 */
p->name = strdup("Alice");
/* 2 */
printf("%s: %d\n", p->name, p->score);
free(/* 3 */);
/* 4 */
  
A. p = malloc(sizeof *p);        B. p = calloc(1, sizeof *p);
C. p = malloc(sizeof(Player));   D. p->score = 0;
E. free(p->name);                F. free(p);
  \end{lstlisting}

  \item[(c)] (4 pts) Write the prototype for a function \texttt{apply} that takes an array of \texttt{double} of length \texttt{n} and a function pointer \texttt{double→double}, and returns a newly allocated \texttt{double*} with transformed values.

  \item[(d)] (3 pts) Explain what happens if you call \texttt{free} twice on the same pointer. Why is this dangerous?

  \item[(e)] (3 pts) Given this typedef:
  \begin{lstlisting}[language=C]
typedef void (*op_fn)(int*, int);
  \end{lstlisting}
  Write a function \texttt{map\_array} that takes an \texttt{int* arr}, its length \texttt{n}, and an \texttt{op\_fn} to apply to each element in place.
\end{enumerate}

%% Problem 2
\section*{Problem 2: Algorithms (20 points)}
\begin{enumerate}
  \item[(a)] (6 pts) For each snippet, give the tightest Big-O in terms of \(n\):\\
  \begin{enumerate}
    \item \texttt{for(int i=0;i<n;i++) for(int j=i+1;j<n;j++) work();}
    \item \texttt{int i=1; while(i<n){ work(); i*=2; }}
    \item \texttt{qsort(a,n,sizeof *a,cmp);}
    \item \texttt{linear\_search(a,n,key);}
    \item \texttt{merge\_sorted(x,m,y,m);}
  \end{enumerate}

  \item[(b)] (6 pts) Describe in-place partitioning in quicksort, and why worst-case is \(\Theta(n^2)\).

  \item[(c)] (8 pts) Sorted array sum-to-\(T\) in \(\Theta(n)\): give two-pointer pseudocode and explain correctness.
\end{enumerate}

%% Problem 3
\section*{Problem 3: Arrays, Array Lists, Linked Lists (20 points)}
\begin{enumerate}
  \item[(a)] (8 pts) Table: for each operation, fill in \(\Theta\)-time for (i) fixed C array, (ii) doubling array list, (iii) singly-linked list with head only.\\
  \begin{tabular}{l|c|c|c}
    Operation               & Array & Array List & Singly-Linked \\
    \hline
    Append at end           &       &            &               \\
    Remove at index \(i\)   &       &            &               \\
    Get element at index \(i\)&    &            &               \\
    Insert at front         &       &            &               \\
  \end{tabular}

  \item[(b)] (6 pts) Draw the memory diagram after:
  \begin{lstlisting}[language=C]
int *a = malloc(4*sizeof *a);
for(int i=0;i<4;i++) a[i]=i+1;
int *b = a+2;
int  c[4];
for(int i=0;i<4;i++) c[i] = a[i]*2;
  \end{lstlisting}

  \item[(c)] (6 pts) Write \texttt{push\_front} and \texttt{pop\_front} for a singly-linked list of \texttt{int} with global \texttt{node* head}.
\end{enumerate}

%% Problem 4
\section*{Problem 4: Trees (15 points)}
\begin{enumerate}
  \item[(a)] (5 pts) Draw BST inserting \(M, C, T, A, J, P, X\) in that order.
  \item[(b)] (5 pts) What insertion order yields minimum-height? What yields maximum-height?
  \item[(c)] (5 pts) Complete this C function for tree height:
  \begin{lstlisting}[language=C]
int tree_height(node *r) {
  if(r==NULL) return 0;
  int hl = /*1*/;
  int hr = /*2*/;
  return /*3*/;
}
  \end{lstlisting}
  Fill in (1)--(3).
\end{enumerate}

%% Problem 5
\section*{Problem 5: Stacks, Queues, Hashtables (15 points)}
\begin{enumerate}
  \item[(a)] (6 pts) For each ADT, state \(\Theta\)-time of core ops: stack (push/pop), queue (enqueue/dequeue), hashtable (insert/lookup).
  \item[(b)] (4 pts) Given ring-buffer queue:
  \begin{quote}
  capacity=8, head=5, tail=2\\
  indices: 0\;1\;2\;3\;4\;5\;6\;7\\
  array: $[,,X,,,A,B,C,_]$
  \end{quote}
  Show state after: enqueue D; dequeue twice; enqueue E, F.
  \item[(c)] (5 pts) Implement \texttt{bool contains\_cycle(node* head);} using tortoise and hare.
\end{enumerate}

%% Problem 6
\section*{Problem 6: Practice Exam Review (10 points)}
\begin{enumerate}
  \item[(a)] (5 pts) For snippets below, write \texttt{INVALID}, \texttt{UNDEFINED}, or the output. Assume variables set up as in class examples.
  \begin{enumerate}
    \item \texttt{b++; printf("\$\%\ c\\n", b[0]);}
    \item \texttt{(*b)++; printf("\$\% \textbackslash n", a);}
    \item \texttt{c = \&b[0]; printf("\$\%\ s\\n", *c);}
  \end{enumerate}
  \item[(b)] (5 pts) For each data structure below, give the asymptotic time of \texttt{remove\_incoming(v)} in a directed graph with \(V\) vertices and \(E\) edges: (i) adjacency matrix, (ii) adjacency list.
\end{enumerate}

%% Problem 7
\section*{Problem 7: ADT Implementation (10 points)}
Complete the C function to remove every other element from a doubly-linked list (head/tail pointers), no leaks:
\begin{lstlisting}[language=C]
void list_thin(list *l) {
  list_node *n = l->head;
  while(n && n->next) {
    list_node *n2 = n->next->next;
    /* A: free second node */
    /* B: relink pointers to skip freed node */
    if(l->tail == n->next) l->tail = n;
    n = n2;
  }
}
\end{lstlisting}
Fill in A and B with proper statements.

%% Problem 8
\section*{Problem 8: Graphs (15 points)}
\begin{enumerate}
  \item[(a)] (7 pts) True/False: Incidence graph of the Fano plane is nonplanar because it contains a subdivision of \(K_{3,3}\). Justify or sketch.
  \item[(b)] (8 pts) Write Dijkstra’s algorithm in pseudocode, showing initialization, relax, and priority queue ops.
\end{enumerate}

\vfill
\noindent\textbf{Good luck!}
\end{document}
